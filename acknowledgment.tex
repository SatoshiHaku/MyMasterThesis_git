
\pagestyle{empty}
\chapter*{謝辞}

\par
本研究の遂行にあたり、大変多くの方々に御指導・御協力を頂きました。心より感謝致します。

慶應義塾大学理工学部物理情報工学科 安藤和也准教授には本研究を遂行する機会を与えて頂いたばかりでなく、研究生活全般にわたり格別な御指導・御助言を頂きました。充実した素晴らしい研究環境を与えて頂きましたことを、またご多忙中にありましても常に気をかけていただきましたことを心より深く感謝申し上げます。



慶應義塾大学理工学部物理情報工学科 牧英之准教授には、本研究を遂行するにあたり不可欠であった実験装置の使用を快く認めて頂いたばかりでなく、本研究の遂行に必要な微細加工についての貴重な御助言を頂きました。心より深く感謝致します。

慶應義塾大学理工学部物理情報工学科 伊藤公平教授には、本研究を遂行する上で必要不可欠なケミカルエリアや実験器具の使用を快く認めて頂きました.心より深く感謝致します。

研究グループの仲間として共に研究を進めた中山裕康博士、田代隆治さん、安紅雨さん、松本貴彦さん、
四谷晋太郎
さん、野村晶代さん、立野裕真
さん、Mohamed Amine Wahadaさん,深見 柾也さん,松浦 早希さん,菅野裕介さん、
桑原勇作さん、江西渚さん、
竹政理恵子さん
には多くの面で御支援を頂きました。心より深く感謝致します。特に
中山裕康博士には本研究に際して多くのご助言とご協力をいただきました.私の拙い実験遂行にお付き合いいただき,さらにディスカッションをしていただきながら研究における理論的な考えや研究に取り組む姿勢を学ばさせていただきました.
また田代隆治さんとは貴重な議論を交わすことができただけでなく,研究の進め方や様々な装置の使い方など多くの有益なご鞭撻をいただきました.

多くの方々にとって不出来な生徒で後輩だったと思いますが,本研究室で研究ができたことを本当に嬉しく思います.
\\
\par
最後に、本研究を行う上で常に支えてくださった家族、友人をはじめとする全ての方々に心より感謝致します。
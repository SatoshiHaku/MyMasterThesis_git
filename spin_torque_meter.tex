
\chapter{スピンホール効果を用いた磁化ダイナミクス変調}\label{spintroquemeter}

強磁性/常磁性金属複合薄膜において、磁化ダイナミクスにより生成されたスピン流が常磁性金属層で吸収されることで、強磁性金属層の磁化歳差運動の緩和が増大した。
これは常磁性金属層におけるスピン流の吸収により、強磁性層のスピン角運動量散逸が増大したことに起因する。この逆過程を考えると、相反性は強磁性体へのスピン流注入により磁化歳差運動緩和の制御が可能であることを示唆している。スピン緩和はスピンの応答性やコヒーレンスを支配するパラメータであり、磁性体の応用上重要な役割を果たす~\cite{Chikazumi,Zutic,Kane}。このため緩和定数を外部から制御する手法が以前より強く求められていたが、物質の種類や温度によって決まる物質パラメータであるため、外部から操作することは困難であった。本研究ではスピンホール効果を用いることでマクロ領域のスピン緩和を電気的に制御可能であることを見出し、スピン流を用いることで磁性体の物性そのものの制御が可能であることを明らかにした。実験結果は
スピントルクを取り入れたLandau-Lifshitz-Gilbert方程式に基づく模型により定量的に再現され、さらにスピン流注入による緩和変調現象はミクロパラメータの仮定なしにスピン流の定量測定を可能にする「スピントルクメーター」としての機能を実現することを明らかにした。



\section{スピンホール効果を用いたスピン緩和変調}
\subsection{Ni$_{81}$Fe$_{19}$/Pt薄膜における磁化ダイナミクス測定}

強磁性共鳴によるマイクロ波吸収スペクトルを解析することで、スピン流注入により変調された磁化ダイナミクスを系統的に調べた。試料として用いたのは
図\ref{relax_sample}(a)に示すNi$_{81}$Fe$_{19}$/Pt薄膜であり、Pt層に電流を流しながら電子スピン共鳴装置を用いた磁化ダイナミクス測定を行った。
Ni$_{81}$Fe$_{19}$層、Pt層の膜厚はそれぞれ10 nmとした。
Ni$_{81}$Fe$_{19}$/Pt薄膜試料を石英菅に装着し、Pt層の端子部分にドータイトを用いてポリウレタン導線を配線した。
その際電子スピン共鳴装置の変調磁場による誘導起電力の影響を除くため、導線を十分にねじった。
このように配線した試料を電子スピン共鳴装置の空洞共振器へ挿入し、直流電流源を用いてPt層に電流を流しながら外部直流磁場を薄膜面内に印加して強磁性共鳴測定を行った。Ni$_{81}$Fe$_{19}$は軟強磁性体であり結晶磁気異方性が極めて小さいため、強磁性共鳴による磁化歳差運動は外部磁場方向を軸とした軌道を描く。
Pt層に電流を流すことでスピンホール効果経由でNi$_{81}$Fe$_{81}$層へスピン流が注入され、このスピン流が磁化ダイナミクスと相互作用する。特に磁化ダイナミクスがスピン流を生成する反作用で磁化歳差運動の緩和が増大したことから、相反性よりスピン流注入によるスピン緩和変調が期待される。





\begin{figure}[htbp]
 \begin{center}
  %\includegraphics[width=90mm]{relax_sample.eps}
\vskip -\lastskip \vskip -3pt
 \caption{試料の模式図と強磁性共鳴スペクトル。
(a) Ni$_{81}$Fe$_{19}$/Pt薄膜の模式図。$\bm{H}$は外部直流磁場、$\bm{J}_c$は電流であり、$\theta$は磁場と電流のなす角を表す。$\theta=0$と$90^\circ$の場合について、$\bm{J}_c$を反転させたときの強磁性共鳴スペクトルの変化を調べた。(b) Ni$_{81}$Fe$_{19}$/Pt薄膜、Ni$_{81}$Fe$_{19}$/Cu薄膜、及びNi$_{81}$Fe$_{19}$薄膜における強磁性共鳴スペクトル$dI(H)/dH$の比較。$I$はマイクロ波吸収強度である。挿入図はスペクトル線幅$W$と強度$S$の定義。(c) スピンホール効果とスピントルクによる緩和変調の模式図。$\bm{M}$, $\bm{H}$はそれぞれ
磁化、外部磁場を表し、$\bm{J}_c$, $\bm{J}_s$, $\bm{\sigma}$はそれぞれ電流、スピン流の空間成分、スピン流のスピン偏極ベクトルを表す。$\bm{n}$はNi$_{81}$Fe$_{19}$/Pt薄膜界面の法線ベクトルである。
}
 \label{relax_sample} 
\end{center}
\end{figure}




本測定ではスピンホール効果により生成されるスピン流のスピン偏極方向が重要となる。
スピンホール効果によって生成されるスピン偏極ベクトル$\bm{\sigma}$は電流$\bm{J}_c$及びスピン流の空間成分$\bm{J}_s$と互いに直交し、
\begin{equation}
\bm{\sigma} \parallel  \bm{J}_c\times\bm{J}_s
\end{equation}
の関係がある。Ni$_{81}$Fe$_{19}$/Pt薄膜において、Pt層に流れる電流によりスピンホール効果経由でNi$_{81}$Fe$_{19}$層に注入されるスピン流の空間成分$\bm{J}_s$は薄膜面法線方向(図\ref{relax_sample}(c)の$\bm{n}$方向)であるため\footnote{本測定で用いたNi$_{81}$Fe$_{19}$/Pt薄膜において、Ni$_{81}$Fe$_{19}$層の一辺はスピン拡散長の長さスケール$\sim 3$ nmと比較して十分大きいため。}、電流と外部磁場を直交させた場合、スピンホール効果によるスピン流のスピン偏極ベクトルと磁化の歳差軸の相対関係(平行・反平行)を電流の反転により切り替えることができる。一方、電流と外部磁場を平行にした場合、電流を反転してもスピン流のスピン偏極ベクトルと磁化の歳差軸の関係は変わらず、
常に直交する。
そこで本測定では電流$\bm{J}_c$と外部磁場$\bm{H}$が平行($\theta=0$)、垂直($\theta=90^\circ$)となる2つの条件において電流反転(スピン流のスピン分極反転)に対する強磁性共鳴スペクトルの変化を調べ、スピン流による磁化ダイナミクス変調効果を検討した。
%はじめに電流を0として強磁性共鳴を5回測定した。ジュール熱による試料の温度変化と温度安定を考え、電流の値を2.5 mAに変えてから10分待ち、その後で強磁性共鳴を5回測定した。さらに電流の向きを反転させ$-2.5$ mAとし、同様の測定を行った。このように電流値を変更し、待ち時間10分として強磁性共鳴を5回測定するサイクルを2.5 mA間隔で20 mAまで繰り返した。
強磁性共鳴測定には周波数 $f=$9.441 GHz、強度 50 mWのマイクロ波を用いた。



図\ref{relax_sample}(b)にNi$_{81}$Fe$_{19}$/Pt薄膜、Ni$_{81}$Fe$_{19}$薄膜及びNi$_{81}$Fe$_{19}$/Cu薄膜の強磁性共鳴スペクトルを示す。また本研究におけるスペクトル線幅$W$及びスペクトル強度$S$の定義を図\ref{relax_sample}(b)の挿入図に示した。スピンポンピングによる緩和の増大、即ち強磁性共鳴スペクトル線幅$W$の増大は常磁性金属層におけるスピン流の吸収(緩和)に起因する。金属におけるスピン流緩和の主要な起源はスピン軌道相互作用である~\cite{Mizukami}。従って常磁性金属の接合によるスペクトル線幅$W$の増大はスピン軌道相互作用の強さの目安となる。図\ref{relax_sample}(b)においてNi$_{81}$Fe$_{19}$/Cu薄膜のスペクトル線幅はNi$_{81}$Fe$_{19}$薄膜の線幅と同程度であり、Cuの弱いスピン軌道相互作用のためにスピン緩和が小さく、強いスピン蓄積が生じていることを示している。一方でNi$_{81}$Fe$_{19}$/Pt薄膜における明瞭なスペクトル線幅の広がりは、Pt層における強いスピン緩和を表し、本測定で用いたNi$_{81}$Fe$_{19}$/Pt薄膜のPt層の強いスピン軌道相互作用を示している。





%測定はPt層に電流を流しながら強磁性共鳴測定を行った。


\begin{figure}[htbp]
 \begin{center}
%\includegraphics[width=85mm]{FMR_90deg.eps}
 \caption{強磁性共鳴スペクトルとスペクトル強度$S$及び線幅$W$の電流$J_c$依存性。
(a) 外部磁場と電流が平行な条件($\theta=0$)で測定したNi$_{81}$Fe$_{19}$/Pt薄膜における強磁性共鳴スペクトル。$I$はマイクロ波吸収強度。
挿入図はスペクトルのピーク付近の拡大図であり、実線と点線はそれぞれ電流反転前後の強磁性共鳴スペクトルである。(b) 外部磁場と電流が垂直な条件($\theta=90^\circ$)で測定したNi$_{81}$Fe$_{19}$/Pt薄膜における強磁性共鳴スペクトル。
(c) 外部磁場と電流が平行な条件($\theta=0$)におけるスペクトル強度$S$とスペクトル線幅$W$の電流$J_c$依存性。$S^*(J_c)\equiv S(J_c)/S(0)$及び$W^*(J_c)\equiv W(J_c)/W(0)$であり、
それぞれ$J_c=0$の場合の強度と線幅で規格化した強度、線幅である。(d) 外部磁場と電流が垂直な条件($\theta=90^\circ$)におけるスペクトル強度$S$とスペクトル線幅$W$の電流$J_c$依存性。
}
 \label{FMR_90deg} 
\end{center}
\end{figure}



電流$\bm{J}_c$及び$-\bm{J}_c$をPt層に流しながら測定した強磁性共鳴スペクトルを図\ref{FMR_90deg}(a)、\ref{FMR_90deg}(b)に示す。
図\ref{FMR_90deg}(a)、\ref{FMR_90deg}(b)はそれぞれ電流と外部磁場が平行となる条件($\theta=0$)、垂直となる条件($\theta=90^\circ$)で測定した結果である。
%ジュール熱による試料温度変化の影響が少ない電流方向反転前後のスペクトルを示した。
本測定で得られたこれら全ての強磁性共鳴スペクトルはローレンツ関数の微分形で再現された。例として、$\theta=90^\circ$, $J_c=0$における強磁性共鳴スペクトルをローレンツ関数の微分形でフィッティングした結果を図\ref{FMR_fit}に示す。このように測定された強磁性共鳴スペクトルがローレンツ関数であることは、薄膜試料の界面や表面の粗雑さに由来する不均一磁場によるスペクトル形状への影響が極めて小さいことを示している。
%強磁性共鳴スペクトルのスペクトル線幅と緩和定数$\alpha$は比例関係にあると考えることができる。


\begin{figure}[tbp]
 \begin{center}
  %\includegraphics[width=68mm]{FMR_fit.eps}
 \caption{Ni$_{81}$Fe$_{19}$/Pt薄膜における$\theta=90^\circ$, $J_c=0$での強磁性共鳴スペクトルをローレンツ関数の微分形でフィッティングした結果。
白丸は測定結果であり、赤線はローレンツ関数の微分形によるフィッティング結果である。}
 \label{FMR_fit}
 \end{center}
\end{figure}





\subsection{電流反転に対する強磁性共鳴スペクトルの対称変化成分}


図\ref{FMR_90deg}(a)、\ref{FMR_90deg}(b)の強磁性共鳴スペクトルは電流値の増大に従い共鳴磁場が高磁場へシフトし、強度$S$が減少することを示している。
さらにスペクトルを詳細に解析した結果、スペクトル線幅$W$は電流の増大に従って増大することがわかった。
強度$S^*$及び線幅$W^*$の電流依存性を図\ref{FMR_90deg}(c)、\ref{FMR_90deg}(d)に示す。ここで$S^*(J_c)\equiv S(J_c)/S(0)$と
$W^*(J_c)\equiv W(J_c)/W(0)$であり、それぞれ$J_c=0$におけるスペクトル強度$S(0)$及び線幅$W(0)$で規格化した強度及び線幅である。
図\ref{FMR_90deg}(c)、\ref{FMR_90deg}(d)のスペクトル変化は電流反転に対して対称であり、電流の流れる方向には依存せず電流の絶対値のみに依存する変化である。










このような電流方向に依存しないスペクトル変化はジュール熱による効果である。
試料に流れる電流の増大はNi$_{81}$Fe$_{19}$層の磁化の熱揺らぎを増幅させる。
熱揺らぎの増大はスペクトル強度$S$を減少させ、スペクトル線幅$W$を増大する~\cite{Bhagat}。このようなジュール熱による熱揺らぎの増大は飽和磁化を減少させるが、この飽和磁化の減少は共鳴磁場のシフト量から定量的に見積もることが可能である。
強磁性薄膜面内に磁場を印加した場合の強磁性共鳴条件は式(\ref{90deg})であるため、共鳴磁場$H_\text{FMR}$は
\begin{equation}
H_\text{FMR}=\frac{1}{2}\left( -4\pi M_s+\sqrt{4\left(\frac{\omega}{\gamma}\right)^2+(4\pi M_s)^2} \right) \label{resonancecondition22}
\end{equation}
で与えられる。図\ref{res_M}に式(\ref{resonancecondition22})から$\omega/\gamma=0.319$ Tとして求めた共鳴磁場の飽和磁化依存性を示した。
図\ref{res_M}が示すように、電流の増大に従って$M_s$が減少すると共鳴条件を満たすために$H_\text{FMR}$は増大する。
これは図\ref{FMR_90deg}(a), \ref{FMR_90deg}(b)の強磁性共鳴スペクトルの共鳴磁場の振る舞いと一致する。
本測定では$J_c=0$において$H_\text{FMR}\approx 134$ mTであり、$J_c=20$ mAにおいて$H_\text{FMR}\approx 137$ mTであったことから、
$J_c=0$で$4\pi M_s=0.628$ Tであった飽和磁化が、$J_c=20$ mAで$4\pi M_s=0.607$ Tに減少したと見積もられる。





\begin{figure}[tbp]
 \begin{center}
  %\includegraphics[width=75mm]{res_M.eps}
  \vskip -\lastskip \vskip -3pt
 \caption{式(\ref{resonancecondition22})において$\omega/\gamma=0.319$ Tとして求めた共鳴磁場$H_\text{FMR}$の飽和磁化$M_s$依存性。
}
 \label{res_M}
 \end{center}
\end{figure}




%\textcolor{red}{
%次にスペクトル強度について考える。(\ref{chichi})式より
%\begin{equation}
%\chi'' =\frac{\gamma M_s\omega\alpha[\omega_0^2+\omega^2(1+\alpha^2)]}{[\omega_0^2-\omega^2(1+\alpha^2)]^2+4\omega^2\omega_0^2\alpha^2}
%\end{equation}
%であるため
%ジュール熱により磁化$M_s$が小さくなると$\chi''$も小さくなる。(\ref{absorption})式に示したとおり、強磁性共鳴によるマイクロ波の吸収強度は$\chi''$に比例するため、電流の増大に従って強磁性共鳴スペクトルの強度が小さくなったと考えられる。
%}


%\textcolor{red}{
%最後にジュール熱によるスペクトル線幅への影響を考える。強磁性共鳴スペクトルの線幅は(\ref{peakwidtha})式より歳差運動の緩和時間$\tau$と反比例の関係にある。即ち、電流の増大による線幅の広がりは歳差運動の緩和時間が短くなったことを意味する。磁化$\bm{M}$が歳差運動しているという状況は、物質中に存在する局在スピンがコヒーレントに歳差運動していることを意味する。局在スピンが感じる内部磁場は完全に同一ではなくそれぞれ異なるため、コヒーレントに歳差運動していた位相は次第にずれ始める。その結果、局在スピンの集合をマクロに見た磁化$\bm{M}$の歳差運動は緩和する(横緩和)。また別の緩和要因としてエネルギーの散逸も同時に生じる(縦緩和)。電流を流すことによってジュール熱が生じると、熱揺らぎによりこれらの緩和が増大する。従って電流の増大によるスペクトル線幅の増大も、ジュール熱による効果であるといえる。以上のように、電流反転に対し対称なスペクトルの変化はすべてジュール熱に由来するものである。スピンホール効果による磁化ダイナミクス変調は、スピン流のスピン分極、言い換えれば電流の方向に依存する。そこで次に電流反転に対して非対称に変化するスペクトルの成分を調べる。
%}



\subsection{電流反転に対する強磁性共鳴スペクトルの非対称変化成分}

電流と磁場が垂直な条件($\theta=90^\circ$)におけるスペクトル強度$S^*(J_c)$と線幅$W^*(J_c)$の電流依存性、図\ref{FMR_90deg}(d)には、小さいが非対称成分が確認できる。この非対称成分は電流反転前後の強磁性共鳴スペクトルを比較することで調べることが可能である。
図\ref{FMR_90deg}(a)、\ref{FMR_90deg}(b)の挿入図に強磁性共鳴スペクトルのピーク付近の拡大図を示した。実線と点線はそれぞれ電流反転前後のスペクトルである。
注目すべきは電流と磁場が垂直な条件($\theta=90^\circ$)で測定した図\ref{FMR_90deg}(b)において、
電流反転に伴ってスペクトルが大きく変化している点である。
これは電流と磁場が平行な条件($\theta=0$)である図\ref{FMR_90deg}(a)には見られない変化である。
前述の通りジュール熱によるスペクトルの変化は電流方向に依存しない。従って$\theta=90^\circ$で検出された電流方向に依存したスペクトルの変化は熱効果以外の機構に起因するものである。






\begin{figure}[tbp]
 \begin{center}
  %\includegraphics[width=120mm]{FMR_asy_90deg.eps}
  \vskip -\lastskip \vskip -3pt
 \caption{(a) 電流と磁場を平行とした場合($\theta=0$)のNi$_{81}$Fe$_{19}$/Pt薄膜における電流反転に対する強磁性共鳴スペクトルの非対称成分$dI(J_c)/dH-dI(-J_c)/dH$。$I$はマイクロ波吸収強度である。$H$は外部直流磁場、$J_c$は電流である。挿入図は強磁性共鳴スペクトルを積分し、非対称成分を調べた結果$I(J_c)-I(-J_c)$である。$H_\text{FMR}$は共鳴磁場を表す。(b) 電流と磁場を垂直とした場合($\theta=90^\circ$)のNi$_{81}$Fe$_{19}$/Pt薄膜における電流反転に対する強磁性共鳴スペクトルの非対称成分$dI(J_c)/dH-dI(-J_c)/dH$。}
 \label{FMR_asy_90deg}
 \end{center}
\end{figure}

電流に依存したスペクトルの非対称な変化を明示するため、図\ref{FMR_asy_90deg}(a)、\ref{FMR_asy_90deg}(b)に電流反転に対する強磁性共鳴スペクトルの非対称成分$dI(J_c)/dH-dI(-J_c)/dH$を示した。
図\ref{FMR_asy_90deg}(a)の電流と磁場が平行な場合($\theta=0$)にはジュール熱に起因する小さな非対称成分しかみられないのに対し、図\ref{FMR_asy_90deg}(b)の
電流と磁場が垂直な場合($\theta=90^\circ$)には明瞭な非対称成分を確認することができる。この結果は電流と磁場が垂直な条件において、電流反転に伴いスペクトル形状が大きく変化したことを明確に示している。


\subsection{スピン緩和の電気的制御とスピンホール効果}


\begin{figure}[tbp]
 \begin{center}
  %\includegraphics[width=120mm]{width_Ptt.eps}
  \vskip -\lastskip \vskip -3pt
 \caption{(a) Ni$_{81}$Fe$_{19}$/Pt薄膜における電流反転に対する強磁性共鳴スペクトルの線幅$W$の非対称成分
$W^*(J_c)-W^*(-J_c)$。$W^*(J_c)\equiv W(J_c)/W(0)$であり、
$J_c=0$におけるスペクトル線幅で規格化した線幅である。%線幅から求めた緩和定数$\alpha$の非対称成分$\alpha(J_c)-\alpha(-J_c)$も示した。
赤線は$\theta=90^\circ$の測定値に関する線形フィッティング結果であり、青線は$W^*(J_c)-W^*(-J_c)=0$を表す。
電流と磁場を平行とした$\theta=0$の場合(青、緑)と電流と磁場を垂直とした$\theta=90^\circ$の場合(黒、赤)にあるそれぞれ2つのプロットは別の試料で測定したものである。
挿入図に示したのは強磁性共鳴スペクトル強度の電流反転に対する非対称成分$S^*(J_c)-S^*(-J_c)$。
$S^*(J_c)\equiv S(J_c)/S(0)$であり、$J_c=0$における強度で規格化した強度である。$\theta=0$において確認される$S^*(J_c)-S^*(-J_c)$の小さな変化は熱効果に由来する。即ち、スペクトルを取得するのに1 min程度の時間が必要とされるため、電流反転前後の測定結果で試料温度が僅かに変化しているためである。これは$\theta=90^\circ$の場合でも同様であるが、$S^*(J_c)-S^*(-J_c)$の変化が$\theta=0$の場合と逆符号であることは、$\theta=90^\circ$に見られるの変化が熱効果によるものではないことを表している。(b) 電流と磁場を垂直とした場合($\theta=90^\circ$)のNi$_{81}$Fe$_{19}$/Cu薄膜及びNi$_{81}$Fe$_{19}$薄膜における電流反転に対する強磁性共鳴スペクトルの線幅$W$の非対称成分
$W^*(J_c)-W^*(-J_c)$。}
 \label{width_Pt}
 \end{center}
\end{figure}

電流と磁場が垂直な条件($\theta=90^\circ$)で観測された電流反転に対する強磁性共鳴スペクトルの非対称な変化は、
電流方向に依存したスペクトル線幅$W$の変化を示唆している。図\ref{width_Pt}(a)に電流反転に対するスペクトル線幅$W$の非対称成分
$W^*(J_c)-W^*(-J_c)$を示す。ここで$W^*(J_c)\equiv W(J_c)/W(0)$であり、
$J_c=0$におけるスペクトル線幅で規格化した線幅である。電流と磁場が垂直な$\theta=90^\circ$において、スペクトル線幅の非対称成分$W^*(J_c)-W^*(-J_c)$は電流に比例して増大した。強磁性共鳴スペクトル線幅$W$、
緩和定数$\alpha$及び磁化歳差運動の緩和時間$\tau$の間には式(\ref{Wtoa})に示したように
 \begin{equation}
W=\frac{2\omega}{\sqrt{3}\gamma}\alpha =\frac{2}{\sqrt{3}\tau\gamma}\label{wa}
\end{equation}
の関係がある。従って図\ref{width_Pt}(a)は$\theta=90^\circ$においてスピン緩和定数$\alpha$の電気的変調が実現されたことを示している。
$J_c=0$において得られたスペクトル線幅$W(0)=7.39$ mTと図\ref{width_Pt}(a)の$W^*(J_c)-W^*(-J_c)$を用いて式(\ref{wa})から求めた$\alpha$の変化量を図\ref{width_Pt3}に示す。

図\ref{width_Pt}(a)の挿入図に示したスペクトル強度$S$の非対称成分$S^*(J_c)-S^*(-J_c)$はスペクトル線幅の非対称成分$W^*(J_c)-W^*(-J_c)$と逆符号の変化を示した。
この電流方向に依存したスペクトル強度$S$の変化も緩和定数$\alpha$の変調に起因する。
式(\ref{strength})に示したように、強磁性共鳴スペクトル強度$S$は$1/\alpha^2$に比例する。
%これはマイクロ波の吸収強度が$\chi''$に比例するため図\ref{FMR}(a)の$I$が$1/\alpha$に比例し、さらにその微分形である図\ref{FMR}(b)の$W$が$\alpha$に比例することを考えれば明らかである\footnote{$W$と$S$の積が$I$に比例するため。}。
強磁性/常磁性金属膜において$J_c=0$における緩和定数を$\alpha_0$とし、電流を流すことで$\Delta \alpha$だけ変化した場合を考える。$J_c$の電流により$\alpha=\alpha_0+\Delta\alpha$, $-J_c$の電流により$\alpha=\alpha_0-\Delta\alpha$と緩和定数が変化したとすると、$\Delta\alpha/\alpha_0\ll 1$のもとで
\begin{equation}
S(J_c)\propto  (\alpha_0+\Delta\alpha)^{-2}\approx \alpha_0^{-2}\left(1-2\frac{\Delta \alpha}{\alpha_0} \right)
\end{equation}
である。同様に
\begin{equation}
S(-J_c)\propto \alpha_0^{-2}\left(1+2\frac{\Delta \alpha}{\alpha_0} \right)
\end{equation}
であり、$S(0)\propto \alpha_0^{-2}$を用いれば、スペクトル強度の非対称成分
\begin{equation}
S^*(J_c)-S^*(-J_c)=\frac{S(J_c)-S(-J_c)}{S(0)}=-4\frac{\Delta \alpha}{\alpha_0} 
\end{equation}
が得られる。一方スペクトル線幅に関しては$W\propto \alpha$より$W(J_c)\propto \alpha_0+\Delta\alpha$, $W(-J_c)\propto \alpha_0-\Delta\alpha$であるため
\begin{equation}
W^*(J_c)-W^*(-J_c)=\frac{W(J_c)-W(-J_c)}{W(0)}=2\frac{\Delta \alpha}{\alpha_0} 
\end{equation}
となる。従って$S^*(J_c)-S^*(-J_c)=-2\left( W^*(J_c)-W^*(-J_c)\right)$であり、スペクトル強度の非対称成分$S^*(J_c)-S^*(-J_c)$はスペクトル線幅の非対称成分$W^*(J_c)-W^*(-J_c)$に対し符号が逆で変化量は2倍となる。これは図\ref{width_Pt}(a)の結果と整合しており、スペクトル強度$S$の変化も$\alpha$の変調を示している。


\begin{figure}[tp]
 \begin{center}
  %\includegraphics[width=65mm]{width_Pt3.eps}
  \vskip -\lastskip \vskip -3pt
 \caption{Ni$_{81}$Fe$_{19}$/Pt薄膜における電流反転に対する強磁性共鳴スペクトルの線幅$W$の非対称成分
$W^*(J_c)-W^*(-J_c)$及び$W^*(J_c)-W^*(-J_c)$から求めた緩和定数$\alpha$の非対称成分$\alpha(J_c)-\alpha(-J_c)$。}
 \label{width_Pt3}
 \end{center}
\end{figure}



緩和定数の電気的変調の起源はPt層におけるスピンホール効果によってNi$_{81}$Fe$_{19}$層に注入されたスピン流が磁化と相互作用し、スピン流が持つスピン角運動量を磁化に受け渡すことで、磁化歳差運動の緩和、言い換えればスピン角運動量の散逸量を変化させたことに起因すると考えられる。以下ではこのようなスピンホール効果によるスピン流と磁化との相互作用以外に考えられる緩和変調機構を検討し、本研究で観測されたスピン緩和変調の起源を解明する。



本測定ではNi$_{81}$Fe$_{19}$/Pt薄膜のPt層に電流を流しながら強磁性共鳴測定を行ったが、このときNi$_{81}$Fe$_{19}$/Pt接合界面を介してNi$_{81}$Fe$_{19}$層にも電流が流れる。従ってNi$_{81}$Fe$_{19}$層に流れる電流の強磁性共鳴スペクトルへの影響を考える必要がある。
また、Pt層に流れる電流はエルステッド磁場によりNi$_{81}$Fe$_{19}$層に不均一磁場を生む。
不均一磁場は見かけのスペクトル線幅を広げるので、線幅変調を解析する際にはこの磁場の影響を考える必要がある。またスピンホール効果に起因するものとして、スピン流注入によってNi$_{81}$Fe$_{19}$層の温度が変化した可能性がある。
Ni$_{81}$Fe$_{19}$層のスピン拡散長は3 nm程度\cite{Bass}であるため、本系において注入されたスピン流の大部分はNi$_{81}$Fe$_{19}$層で吸収されている。
スピン流が緩和する際、スピン流の持つエントロピーが消失するため、スピン流の緩和に伴って温度変化が生じる可能性がある。
前述の通り試料の温度変化はスペクトル線幅を変化させるため、この温度変化により線幅が変調した可能性を議論する必要がある。

以上のようにNi$_{81}$Fe$_{19}$/Pt薄膜に電流を流した場合にスペクトル変調をもたらす要因として考えられるのは以下の4点である。
\begin{enumerate}
\item スピンホール効果によりNi$_{81}$Fe$_{19}$層へ注入されたスピン流と磁化の相互作用
\item Ni$_{81}$Fe$_{19}$層に流れる電流
\item Pt層の電流が作るエルステッド磁場による不均一磁場
\item スピンホール効果を介したスピン流注入によるNi$_{81}$Fe$_{19}$層の温度変化
\end{enumerate}



Ni$_{81}$Fe$_{19}$層に流れる電流及びエルステッド磁場による影響を考えるため、PtをCuに変えたNi$_{81}$Fe$_{19}$/Cu薄膜及びPtをなくしたNi$_{81}$Fe$_{19}$薄膜についても同様の強磁性共鳴測定を行った。
Ni$_{81}$Fe$_{19}$/Pt薄膜でスペクトルの非対称変調が観測された磁場と電流が垂直な条件($\theta=90^\circ$)で測定した結果を図\ref{width_Pt}(b)に示す。図\ref{width_Pt}(b)が示すように、Ni$_{81}$Fe$_{19}$/Cu薄膜及びNi$_{81}$Fe$_{19}$薄膜では
電流と磁場を垂直とした場合でも緩和変調は観測されなかった。この結果はNi$_{81}$Fe$_{19}$層に流れる電流やエルステッド磁場による不均一磁場が緩和変調と無関係であることを意味している。



Ni$_{81}$Fe$_{19}$/Pt薄膜に電流を流した場合のエルステッド磁場を見積もるため、強磁性共鳴磁場$H_\text{FMR}$の電流反転に対する非対称成分$H_\text{FMR}(J_c)-H_\text{FMR}(-J_c)$を図\ref{res_asym}(a)に示した。$J_c=20$ mAにおいて共鳴磁場の非対称成分は0.08 mT程度であり、電流によるエルステッド磁場は共鳴時における外部磁場$\approx 135$ mTと比較して
$0.04/135\approx 3\times10^{-4}$程度となり、本測定における不均一磁場による影響は極めて小さいといえる。 


\begin{figure}[tbp]
 \begin{center}
  %\includegraphics[width=113mm]{res_asym.eps}
 \caption{(a) Ni$_{81}$Fe$_{19}$/Pt薄膜における電流反転に対する共鳴磁場$H_\text{FMR}$の非対称成分$H_\text{FMR}(J_c)-H_\text{FMR}(-J_c)$。Pt層に流れた電流が作るエルステッド磁場は
$d\bm{M}/dt=-\gamma\bm{M}\times\bm{H_\text{eff}}+(\alpha/M_s) \bm{M}\times d\bm{M}/dt $において
$\bm{H_\text{eff}}$に取り入れられるため、共鳴条件を満たすような外部磁場、即ち$H_\text{FMR}$をシフトさせる。黒丸は測定結果であり、青線は線形フィッティング結果を表す。(b) Ni$_{81}$Fe$_{19}$/Pt薄膜における
温度変化。微小熱電対を用いてPt層に電流($5$ mAもしくは$-5$ mA)を流しながらNi$_{81}$Fe$_{19}$層の温度変化を調べた。赤線が試料に流した電流を表す。}
 \label{res_asym}
 \end{center}
\end{figure}




スピン流注入によるNi$_{81}$Fe$_{19}$層の温度変化の影響を調べるため、電流と磁場が垂直な条件でPt層に電流を流しながら微小熱電対を用いて測定したNi$_{81}$Fe$_{19}$層表面の温度変化を図\ref{res_asym}(b)に示す。図\ref{res_asym}(b)は電流の反転とNi$_{81}$Fe$_{19}$層の温度の間に明確な相関を示しておらず、本測定において
スピン流熱効果による寄与は極めて小さいといえる。以上のように、
Ni$_{81}$Fe$_{19}$層を流れる電流、エルステッド磁場による不均一磁場及びスピン注入による温度変化は緩和変調の起源としては不適当である。




Ni$_{81}$Fe$_{19}$/Cu薄膜において緩和変調が生じないことは、
Ni$_{81}$Fe$_{19}$/Pt薄膜で実現されたスピン緩和変調がPt層におけるスピンホール効果に起因することを示している。
図\ref{relax_sample}(b)に示したNi$_{81}$Fe$_{19}$/Pt薄膜及びNi$_{81}$Fe$_{19}$/Cu薄膜における強磁性共鳴スペクトルはPtの接合による明瞭なスピン緩和の増大を示しており、本測定で用いたPtのスピン軌道相互作用がCuと比較して極めて強いことを示す結果である。
従ってCuにおけるスピンホール効果はPtと比較して極めて小さい。さらに電流と磁場の相対関係が示す対称性、即ちNi$_{81}$Fe$_{19}$/Pt薄膜において電流と磁場が平行な条件では緩和変調が実現されないことも、緩和変調の起源がスピンホール効果であることを支持している。


\subsection{スピントルクによるスピン緩和変調効果:スピントルクメーター}




Ni$_{81}$Fe$_{19}$/Pt薄膜で実現された緩和変調はスピンホール効果を経由したスピントルクによるスピン緩和変調の模型と整合する。
緩和変調が実現された電流と磁場が垂直な条件($\theta=90^\circ$)において、スピンホール効果によりNi$_{81}$Fe$_{19}$層に注入されるスピン流のスピン偏極ベクトル$\bm{\sigma}$は
磁化${\bm M}$の歳差運動の軸方向と平行(反平行)である。このスピン流が磁化に与えるスピントルクは図\ref{spinHall_spintorque2}に示すように
磁化歳差運動の間、常に緩和トルク${\bm D}$と平行(反平行)である\footnote{スピントルクの起源は全角運動量保存則、式(\ref{M_hozon})であり、ミクロスコピックには$s-d$相互作用に起因する。}。従ってこのスピントルクが緩和トルクを実効的に増大(減少)させ、
緩和定数$\alpha$を変調する。電流と磁場が平行な条件($\theta=0$)
ではスピントルクは歳差運動1周期で打ち消される。そのため緩和変調が実現されるのは電流と磁場が垂直な場合のみである。
以上のようにNi$_{81}$Fe$_{19}$/Pt薄膜で実現された電気的な緩和変調はスピンホール効果によるスピントルクに起因するものである。


\begin{figure}[htbp]
 \begin{center}
  %\includegraphics[width=68mm]{spinHall_spintorque2.eps}
 \caption{スピントルクによる緩和変調の模式図。$\bm{M}$, $\bm{H}$はそれぞれ
磁化、外部磁場を表す。スピン流が磁化に与えるスピントルクは、磁化が歳差運動している間、常に緩和トルクと反平行のトルクを与え、実効的に緩和を減少させる。Ni$_{81}$Fe$_{19}$/Pt薄膜において電流を反転するとスピン流のスピン分極が反転し、スピン流が磁化に与えるスピントルクも反転するため、緩和トルクと平行なトルクが働き、この場合実効的に緩和は増大する。
}
 \label{spinHall_spintorque2}
 \end{center}
\end{figure}




スピントルクを取り入れたLLG方程式に基づきスピン流注入によるスピン緩和変調の現象論的な模型を構築する~\cite{AndoPRL}。
Ni$_{81}$Fe$_{19}$/Pt薄膜における磁化のダイナミクスはスピンポンピングによる緩和の増大をもたらす。電流を流していない場合($\bm{J}_c=0$)、
スピンポンピングのみが緩和定数に寄与する。スピンポンピングはPt層にスピン流を注入するため、Ni$_{81}$Fe$_{19}$層のスピン角運動量散逸は増大する。即ちスピンポンピングが生じているとき緩和は増大し、Ni$_{81}$Fe$_{19}$層の緩和定数$\alpha$はNi$_{81}$Fe$_{19}$薄膜の内因的な緩和定数$\alpha_F$と
スピンポンピングによる緩和定数の変化分$\Delta \alpha_\text{SP}$との和、$\alpha=\alpha_F+\Delta \alpha_\text{SP}$となる~\cite{Tserkovnyak1,Mizukami}。


電流が有限の場合($\bm{J}_c\not=0$)、スピンホール効果により生成されたスピン流$J_s$はNi$_{81}$Fe$_{19}$/Pt界面を介してNi$_{81}$Fe$_{19}$層に注入され、Ni$_{81}$Fe$_{19}$層で吸収されることで磁化にスピントルクを与える。電流と磁場が垂直な条件において、スピン流のスピン偏極ベクトル$\bm{\sigma}$は磁化歳差運動の軸方向を向く。この条件ではスピンポンピングによる緩和の増大に加え、スピンホール効果によるスピントルクが緩和変調をもたらす。LLG方程式に式(\ref{spintorque})のスピントルクを取り入れた、一般化されたLLG方程式\cite{Slonczewski}
\begin{equation}
\frac{d\bm{M}}{dt}=-\gamma\bm{M}\times\bm{H}_\text{eff}+\frac{\alpha}{M_s}\bm{M}\times\frac{d\bm{M}}{dt}-\frac{\gamma J_s}{M_s^2 V_F}\bm{M}\times\left(\bm{M}\times\bm{\sigma}\right)\label{LLGspintorque}
\end{equation}
に基づき、スピントルクによるスピン緩和変調現象を記述する。右辺第3項目がスピンホール効果によるスピントルク項である。$V_F$はNi$_{81}$Fe$_{19}$層の体積である。
$z$軸方向に外部磁場を印加し、$x-y$面内で磁化が歳差運動する図\ref{relax_sample}(c)の状況を考える。有効磁場${\bm H}_\text{eff}$として外部磁場、反磁場及びマイクロ波磁場を考える。即ち
\begin{equation}
{{\bm H}_\text{eff}} = \left( {\begin{array}{*{20}{c}}
   0  \\
   0  \\
   H  \\
\end{array}} \right) + \left( {\begin{array}{*{20}{c}}
   0  \\
   { - 4\pi {m_y}(t)}  \\
   0  \\
\end{array}} \right) + \left( {\begin{array}{*{20}{c}}
   h_\text{ac}(t)  \\
   0  \\
   0  \\
\end{array}} \right)\label{HHH}
\end{equation}
とする。また磁場と電流が垂直な状況を考え、Ni$_{81}$Fe$_{19}$層に注入されるスピン流のスピン偏極を$\bm{\sigma}=(0, 0, 1)$とする。

マイクロ波を$h_\text{ac}(t)=h e^{i\omega t}$とし、
磁化歳差運動の振動成分を$m_{x}(t)=m_{x} e^{i\omega t}$及び$m_{y}(t)=m_{y} e^{i\omega t}$とすると、式(\ref{HHH})のもとで式(\ref{LLGspintorque})は
\begin{equation}
\frac{1}{{{M_s}}}\left( {\begin{array}{*{20}{c}}
   {H + i\alpha \dfrac{\omega }{\gamma }} & { - i\dfrac{\omega }{\gamma } - \dfrac{{{J_s}}}{{{M_s}{V_F}}}}  \\
   {i\dfrac{\omega }{\gamma } + \dfrac{{{J_s}}}{{{M_s}{V_F}}}} & {H + 4\pi {M_s} + i\alpha \dfrac{\omega }{\gamma }}  \\
\end{array}} \right)\left( {\begin{array}{*{20}{c}}
   {{m_x}}  \\
   {{m_y}}  \\
\end{array}} \right) = \left( {\begin{array}{*{20}{c}}
   h  \\
   0  \\
\end{array}} \right)
\end{equation}
となる。
従って歳差運動の振動成分
\begin{equation}
\left( {\begin{array}{*{20}{c}}
   {{m_x}}  \\
   {{m_y}}  \\
\end{array}} \right) = {M_s}\frac{{\left( {\begin{array}{*{20}{c}}
   {H + 4\pi {M_s} + i\alpha \dfrac{\omega }{\gamma }} & {i\dfrac{\omega }{\gamma } + \dfrac{{{J_s}}}{{{M_s}{V_F}}}}  \\
   { - i\dfrac{\omega }{\gamma } - \dfrac{{{J_s}}}{{{M_s}{V_F}}}} & {H + i\alpha \dfrac{\omega }{\gamma }}  \\
\end{array}} \right)\left( {\begin{array}{*{20}{c}}
   h  \\
   0  \\
\end{array}} \right)}}{{\left( {H + i\alpha \dfrac{\omega }{\gamma }} \right)\left( {H + 4\pi {M_s} + i\alpha \dfrac{\omega }{\gamma }} \right) + {{\left( {i\dfrac{\omega }{\gamma } + \dfrac{{{J_s}}}{{{M_s}{V_F}}}} \right)}^2}}}\label{014}
\end{equation}
を得る。$\alpha$の2次以上を無視すると、式(\ref{014})より磁化率は
\begin{equation}
\chi_{xx}=\frac{M_s\left(H + 4\pi {M_s} + i\alpha \dfrac{\omega }{\gamma }\right)}{R +2i \dfrac{\omega }{\gamma }(H + 2\pi {M_s})\left( {\alpha  + \dfrac{{{J_s}}}{{{M_s}{V_F}}}\dfrac{1}{{H + 2\pi {M_s}}}} \right)}
\end{equation}
となる。ここで
\begin{equation}
R=H(H + 4\pi {M_s}) - {\left( {\frac{\omega }{\gamma }} \right)^2} + {\left( {\frac{{{J_s}}}{{{M_s}{V_F}}}} \right)^2}\label{resoresocon}
\end{equation}
であり、強磁性共鳴条件は$R=0$である。
%=========================================================
%\begin{equation}
%\chi_{xx}=\frac{M_s\left(H + 4\pi {M_s} + i\alpha \dfrac{\omega }{\gamma }\right)}{H(H + 4\pi {M_s}) - {\left( {\dfrac{\omega }{\gamma }} \right)^2} + {\left( {\dfrac{{{J_s}}}{{{M_s}{V_F}}}} \right)^2} +2i \dfrac{\omega }{\gamma }(H + 2\pi {M_s})\left( {\alpha  + \dfrac{{{J_s}}}{{{M_s}{V_F}}}\dfrac{1}{{H + 2\pi {M_s}}}} \right)}
%\end{equation}
%となる。従って強磁性共鳴条件は
%\begin{equation}
%H(H + 4\pi {M_s}) - {\left( {\frac{\omega }{\gamma }} \right)^2} + {\left( {\frac{{{J_s}}}{{{M_s}{V_F}}}} \right)^2}=0\label{resoresocon}
%\end{equation}
%=========================================================
この解を強磁性共鳴磁場$H=H_\text{FMR}$とすると
\begin{equation}
H_\text{FMR}=-2\pi M_s+\sqrt{\left( \frac{\omega}{\gamma}\right)^2+4\pi^2 M_s^2-\left( {\frac{{{J_s}}}{{{M_s}{V_F}}}} \right)^2}
\end{equation}
であり、共鳴磁場付近$H\simeq H_\text{FMR}$で磁化率$\chi_{xx}=\chi_{xx}'-i\chi_{xx}''$の虚数成分は
\begin{equation}
\chi_{xx}'' = \frac{1}{4\pi}(4\pi {M_s})  \left(\frac{{ {{H_\text{FMR}} + 4\pi {M_s}}  }}{  2H _\text{FMR}+ 4\pi {M_s} }\right)\frac{\alpha_\text{eff}  \left(\dfrac{\omega}{\gamma}\right)}{ { {{(H - {H_\text{FMR}})}^2}+\left( \alpha_\text{eff} \dfrac{\omega}{\gamma}\right)^2 } }
\end{equation}
と表せる\footnote{計算方法の詳細は第\ref{formulation}章参照。}。ここで$\alpha_\text{eff}$がスピントルクによって変調された緩和定数であり、
スピントルクによる緩和変調量を$\Delta \alpha_s$とすると
\begin{equation}
\alpha_\text{eff}=\alpha  + \Delta\alpha_s=\alpha+\frac{{{J_s}}}{{{M_s}{V_F}}}\frac{1}{{H + 2\pi {M_s}}}
\end{equation}
である。従って共鳴磁場付近を考え式(\ref{resoresocon})を用いると、スピントルクによる緩和変調量は
\begin{align}
\Delta\alpha_s&= \frac{{{J_s}}}{{{M_s}{V_F}}}\frac{1}{{H + 2\pi {M_s}}}\nonumber\\
&=\frac{{\gamma {J_s}}}{{\omega {M_s}{V_F}}}\left[1 + {{\left( {\dfrac{{4\pi {M_s}/2}}{{\omega /\gamma }}} \right)}^2} - {\left( {\dfrac{{{J_s}/({M_s}{V_F})}}{{\omega /\gamma }}} \right)}^2\right]^{-1/2} \nonumber\\
&\approx \frac{{\gamma {J_s}}}{{\omega {M_s}{V_F}}}\label{spinmeter}
\end{align} 
となる。ここで$( 4\pi {M_s}/2)/(\omega /\gamma )\approx 1$及び$(J_s/({M_s}{V_F}))/(\omega /\gamma )\ll 1$を用いた。式(\ref{spinmeter})はスピントルクによる緩和変調量$\Delta\alpha_s$が注入されたスピン流量$J_s$に比例し、$\bm{J}_s$の方向により緩和定数の増大($\Delta\alpha_s>0$)・減少($\Delta\alpha_s<0$)が制御可能であることを示している。これはNi$_{81}$Fe$_{19}$/Pt複合薄膜におけるスピンホール効果による緩和変調の実験結果と整合する。


式(\ref{spinmeter})より注入されたスピン流量$J_s$は
\begin{equation}
J_s=\frac{\omega M_s V_F}{\gamma}\Delta \alpha_s\label{spintorquemetereq}
\end{equation}
と書ける。注目すべき点は$\Delta\alpha_s$と$J_s$
が比例関係にあり、さらに比例係数が強磁性共鳴により測定可能なマクロパラメータのみで構成されることである。これはスピン流注入による緩和変調$\Delta \alpha_s$を検出することで、ミクロパラメータの仮定なしに
スピントルク或いはスピン流$J_s$の定量的測定が可能であることを示している。
電流と磁場が垂直な$\theta=90^\circ$の条件において、$J_c=0$における強磁性共鳴スペクトル線幅は$W(0)=7.39$ mTである。
$W(0)$及び$f=9.441$ GHzを用いることで、式(\ref{wa})より緩和定数$\alpha(0)=0.02$が得られる。
また強磁性共鳴磁場から$4\pi M_{\rm s} = 0.628$ Tと見積もられる。パラメータとしてマイクロ波周波数$f=9.441$ GHz、飽和磁化$M_s=0.050$ T、Ni$_{81}$Fe$_{19}$層の体積$V_F=4.8\times 10^{-9}$ cm$^3$、$g$因子$g=2.11$、ボーア磁子$\mu_B=9.27\times10^{-25}$ Tcm$^3$及び$\gamma=g \mu_B/\hbar$を用いると、式(\ref{spintorquemetereq})よりNi$_{81}$Fe$_{19}$層全体に注入されたスピン流は
\begin{equation}
J_s=\frac{2\omega M_s V_F }{g \mu_B}\Delta \alpha_s\left(\frac{\hbar}{2}\right)=1.46\times10^{25}\Delta \alpha_s\left(\frac{\hbar}{2}\right)\label{gegeg}
\end{equation}
となる。図\ref{width_Pt}(a)において$\theta=90^\circ$における線形フィッティング結果(赤線)は$W^*(J_c)-W^*(-J_c)=3.33\times 10^{-1} J_c $である。ここで$J_c$の単位は[A]とした。これより
$J_c=20$ mAにおいて$W^*(20 \;\text{mA})-W^*(-20\; \text{mA})=0.00666$であり、緩和変調量は
\begin{eqnarray}
\Delta \alpha_s(20 \;\text{mA})&=&\frac{\alpha(J_c)-\alpha(-J_c)}{2}= \frac{1}{2}\frac{\sqrt{3}\gamma}{2\omega}\left(W^*(J_c)-W^*(-J_c)\right)
W(0)\nonumber\\
&=&6.68\times 10^{-5}
\end{eqnarray}
である。従って式(\ref{gegeg})より$J_c=20$ mAにおけるスピン流注入量を求めると、$J_s=9.7\times10^{20}$ spins/sが得られる。試料に流した電流$J_c$、線幅変調量$W^*(J_c)-W^*(-J_c)$、Ni$_{81}$Fe$_{19}$層全体に注入されたスピン流$J_s$の関係を図\ref{spinmeterA}に示す。

\begin{figure}[tbp]
 \begin{center}
  %\includegraphics[width=68mm]{spinmeterA.eps}
 \caption{Ni$_{81}$Fe$_{19}$/Pt薄膜における試料に流した電流$J_c$、線幅変調量$W^*(J_c)-W^*(-J_c)$、Ni$_{81}$Fe$_{19}$層全体に注入されたスピン流$J_s$の関係。
}
 \label{spinmeterA}
 \end{center}
\end{figure}


これまでに知られていたスピン流検出技術ではミクロパラメータの仮定が不可欠であった。本研究により見出された緩和変調を用いたスピン流の定量測定「スピントルクメーター」はミクロパラメータの仮定なしにスピン流の定量を可能とする希有な手段であり、スピン流物理の開拓及びスピントロニクスデバイスの拡充に本質的役割を果たすことが期待される。





\subsection{スピントルクメーターを用いたスピンホール角の決定}
スピントルクメーター効果を用いることで、スピンホール角$\theta_\text{SHE}\equiv \sigma_\text{SHE}/\sigma^N_c$を求めることが可能である。ここで$\sigma_\text{SHE}$はスピンホール伝導度、$\sigma_c^N$は常磁性金属層の電気伝導度である。Ni$_{81}$Fe$_{19}$/Pt薄膜に加えNi$_{81}$Fe$_{19}$/Pd薄膜を作成し、Ni$_{81}$Fe$_{19}$/Pt薄膜と同様の強磁性共鳴測定を行った。Ni$_{81}$Fe$_{19}$/Pd薄膜のNi$_{81}$Fe$_{19}$層の膜厚は10 nm、Pd層の膜厚は20 nmとした。図\ref{hallangle}にNi$_{81}$Fe$_{19}$/Pt薄膜及びNi$_{81}$Fe$_{19}$/Pd薄膜における電流反転に対する強磁性共鳴スペクトルの線幅$W$の非対称成分
$W^*(J_c)-W^*(-J_c)$を示す。ここで$j_c$は常磁性金属(Pt, Pd)層に流れた電流密度を等価回路計算により求めたものである。等価回路計算には四端子法により求めたNi$_{81}$Fe$_{19}$、Pt、Pdの電気伝導度を用いた。
Ni$_{81}$Fe$_{19}$/Pt試料において、Ni$_{81}$Fe$_{19}$層の電気伝導度$\sigma_c^F=1.09\times10^6$ $(\Omega \text{m})^{-1}$、Pt層の電気伝導度$\sigma_c^N=1.79\times10^6$ $(\Omega \text{m})^{-1}$であり、Ni$_{81}$Fe$_{19}$/Pd試料において、Ni$_{81}$Fe$_{19}$層の電気伝導度$\sigma_c^F=2.10\times10^6$ $(\Omega \text{m})^{-1}$、Pd層の電気伝導度$\sigma_c^N=3.54\times10^6$ $(\Omega \text{m})^{-1}$であった。Ni$_{81}$Fe$_{19}$/Pd試料も明確なスペクトル線幅の非対称成分を示しており、Pd層におけるスピンホール効果を観測した結果であるといえる。




\begin{figure}[tbp]
 \begin{center}
  %\includegraphics[width=55mm]{hallangle.eps}
 \caption{Ni$_{81}$Fe$_{19}$/Pt薄膜及びNi$_{81}$Fe$_{19}$/Pd薄膜における電流反転に対する強磁性共鳴スペクトルの線幅$W$の非対称成分
$W^*(J_c)-W^*(-J_c)$。$j_c$は常磁性金属(Pt, Pd)層に流れた電流密度を等価回路計算により求めたものである。}
 \label{hallangle}
 \end{center}
\end{figure}


Valet-Fert模型~\cite{Valet,Maekawa}によれば、強磁性/常磁性金属薄膜において、常磁性金属層におけるスピンホール効果により強磁性金属層に注入されるスピン流は
\begin{equation}
J_s=\eta\theta_\text{SHE}A_{F/N}\frac{\hbar}{e}\frac{{J_c}}{A_N}\label{spincurrent22121}
\end{equation}
と表される。ここで$A_{F/N}$は強磁性/常磁性金属界面の面積であり、$A_N$は常磁性金属層の断面積である。$\eta $はスピン注入効率であり、
\begin{equation}
\eta  = \frac{{2\sinh ^2 (d_{N} /2\lambda _{N} ){\rm{/}}\cosh (d_{N} /\lambda _{N} )}}{{1 + (\lambda _{F} {\rm{/}}\lambda _{N} )(\sigma _{\rm{c}}^{N} {\rm{/}}\sigma _{\rm{c}}^{F} )\tanh (d_{N} {\rm{/}}\lambda _{N} )/\tanh (d_{F} {\rm{/}}\lambda _{F} )}}
\end{equation}
である~\cite{AndoPRL}。$d_N$($d_F$)は常磁性金属(強磁性金属)層の膜厚、$\sigma^F_c$は強磁性金属層の電気伝導度、$\lambda_N$($\lambda_F$)は常磁性金属(強磁性金属)層のスピン拡散長である。従って式(\ref{spinmeter})及び(\ref{spincurrent22121})より、スピンホール効果による緩和変調$\Delta \alpha_s$は
\begin{eqnarray}
  \Delta\alpha_s
   \approx
  \left( 
    \frac{\hbar\gamma\eta\theta_\text{SHE}}{2\pi f M_s e A_N d_F} 
  \right)
   J_c
   \label{eq:Dalpha}				
\end{eqnarray}					
となる。



Ni$_{81}$Fe$_{19}$/Pd薄膜における強磁性共鳴測定ではマイクロ波周波数$f=9.436$ GHzであり、測定した強磁性共鳴スペクトルより$W(0)=5.094$ mT、飽和磁化$4\pi M_s=0.784$ Tが得られる。図\ref{hallangle}に示したスペクトル線幅変調の結果よりNi$_{81}$Fe$_{19}$/Pt薄膜について$\eta \theta_\text{SHE}\approx 0.051$、Ni$_{81}$Fe$_{19}$/Pd薄膜について$\eta \theta_\text{SHE}\approx 0.038$となる。電気伝導度及びスピン拡散長$\lambda_\text{NiFe}=3$ nm、$\lambda_\text{Pt}=5$ nm、$\lambda_\text{Pd}=10$ nm~\cite{Bass}を用いることで、Pt層におけるスピンホール角$\theta_\text{SHE} \sim 0.13$、Pd層におけるスピンホール角$\theta_\text{SHE} \sim 0.037$が得られる。
本結果はNi$_{81}$Fe$_{19}$/Pt接合とNi$_{81}$Fe$_{19}$/Pd接合におけるミキシングコンダクタンスが同程度であり、スピンポンピングを駆動した際に生成されるスピン流量が同程度であることを仮定すれば、第\ref{nobl}節において示したスピンポンピングによる逆スピンホール効果の結果と整合するものである。






\section{本章のまとめ}
本章で得られた主要な結果は以下の4点である。
\begin{enumerate}
 \item Ni$_{81}$Fe$_{19}$/Pt複合膜における電気的なスピン緩和変調を実現した。
 \item Ni$_{81}$Fe$_{19}$/Pt複合膜におけるスピン緩和変調はスピンホール効果を介した強磁性層へのスピン流注入が誘起するスピントルクに起因することを明らかにした。
 \item スピントルクを取り入れたLandau-Lifshitz-Gilbert方程式に基づき、Ni$_{81}$Fe$_{19}$/Pt複合膜における緩和変調を定量的に再現する模型を構築した。
 \item スピン流注入による緩和変調測定はミクロパラメータの仮定なしにスピン流の定量測定を可能とする「スピントルクメーター」の機能を実現することを明らかにした。
\end{enumerate}









